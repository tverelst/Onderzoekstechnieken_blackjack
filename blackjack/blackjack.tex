\usepackage{datetime}
\documentclass[journal]{IEEEtran}
\hyphenation{op-tical net-works semi-conduc-tor}
\begin{document}
\title{Kunnen we we het casino verslaan?}
\author{Caudenberg~Samuel, Demeulemeester~Michiel, De~Wilde~Michiel, Verelst~Thomas}
\maketitle


\begin{abstract}
Tijdens ons project Onderzoekstechnieken gaan we op zoek naar de invloed op de winst bij een ronde BlackJack, naargelang het aantal spelers, waarbij 1 speler (die we aanschouwen als onszelf) gebruik maakt van de Card Counting Strategy en alle andere spelers de basic Strategy toepassen. De resultaten van dit onderzoek zullen aantonen wanneer de speler die gebruik maakt van de Card Counting Strategy een relatief hoge kans maakt om de ronde te winnen, of wanneer hij het net bij een kleine bet moet houden. Met behulp van enkele simulaties hebben we de nodige data verzameld, daarna hebben we deze geanalyseerd en uiteindelijk tonen we aan wat we onderzocht hebben door gebruik te maken van grafieken. 
\end{abstract}

\section{Voorwoord}
Om duidelijk te maken welke strategie�n we gaan gebruiken, zullen we deze beiden kort toelichten. 
Eerst hebben we de Basic Strategy (zie onderstaande tabel).

De betekenis van 'Stiff hand': Als de eerste kaarten een totaal geven van 12-16, is dit een Stiff hand'. Wanneer dit voor de speler het geval is, en ook de dealer heeft een Stiff hand (hand tussen 2 en 6), dan moet je 'Standen'. Heeft de dealer echter een 'Pat' ( hand tussen 7 en Ace), dan moet de speler 'Hitten'. 
'Doubling down hands' betekent het verdubbelen van je bet, en dit doet de speler wanneer zijn eerste kaarten een totaal geven tussen 9 en 11 punten.

De tweede strategie die we gebruiken tijdens dit onderzoek is de Card Counting Strategy. 
 
\hfill \today

\section{Aanpak}
\subsection{Strategi�n}
\subsubsection{Count Strategie}
\subsubsection{Betting Strategie}
\subsection{Simulatiesoftware}
\subsection{Uitwerking}

\section{Resultaten}
\subsection{Aantal spelers als variabele}
We bekijken eerst wat de invloed is van het aantal spelers aan de tafel. 
Onze speler gebruikt een countstrategie, terwijl de andere spelers aan zijn tafel een gewone basic strategy toepassen. Het startkrediet is telkens \textdollar 10.000 en er worden telkens 10.000.000 rondes gespeeld. In tabel \ref{table:spelers} wordt het resterende krediet ook in percent ten op zichte van het startkrediet getoond, zodat duidelijk is dat het verschil minimaal is.

\begin{table}[h]
\normalsize
\caption{Krediet na 10.000.000 rondes met \textdollar 10.000 startkrediet}
\label{table:spelers}
\begin{center}
\def\arraystretch{1.5}
\begin{tabular}{c|c c c c}
Aantal spelers & 1 & 2 & 3 & 4\\
\hline
Krediet & \textdollar 96,78 & \textdollar 107,77 & \textdollar 92,82 & \textdollar 111,27 \\
Percent & 0,9678\% & 1,077\% & 0,9282\% & 1,1127\% \\
\end{tabular}
\end{center}
\end{table} 

\subsection{Vergelijking met de andere spelers}
Mits het duidelijk is dat er toch nog verlies wordt gemaakt, vergelijken we nu de resultaten met de resultaten van de andere spelers aan tafel. We maken hier enkel de vergelijking wanneer er vier spelers aan de tafel zitten, want in dat scenario hadden uiteindelijk het minste verlies.

Dezelfe regels gelden als in voorgaande sectie. Speler 1 speelt dus met een count strategie, en de andere spelers aan de tafel met een gewone basic strategy. De spelers beginnen allen met \textdollar 10.000 startkrediet en er worden 10.000.000 rondes gespeeld. Een negatief bedrag betekent dat de speler al zijn startkrediet heeft opgespeeld en dus extra krediet ter beschikking heeft gekregen.

We introduceren in tabel \ref{table:vierspelers} nu ook het begrip \textit{risk of ruin}. Dit is het percentage waarmee aangeduid wordt hoe groot de kans is dat iemand al zijn krediet verliest. De \textit{risk of ruin} hangt dan ook af van hoeveel ingezet wordt en hoeveel keren gespeeld wordt.

\begin{table}[h]
\normalsize
\caption{Vergelijking van de spelers na 10.000.000 rondes met \textdollar 10.000 startkrediet}
\label{table:vierspelers}
\begin{center}
\def\arraystretch{1.5}
\begin{tabular}{c|c c c c}
Speler & 1 & 2 & 3 & 4\\
\hline
Krediet & \textdollar 111,27 & \textdollar -33,24 & \textdollar -35,68 & \textdollar -35,67 \\
Percent & 1,1127\% & -0.3324\% & -0,3568\% & 0,9282\% \\
Risk of ruin & 13,53\% & 100\% & 100\% & 100\% \\
\end{tabular}
\end{center}
\end{table} 

\section{Conclusies}

De conclusies


\begin{thebibliography}{1}
\bibitem{IEEEhowto:kopka}
Ariell Zimran, Anna Klis, Alejandra Fuster and Christopher Rivelli, \emph{The Game of Blackjack and Analysis of Counting Cards }\hskip 1em plus 0.5em minus 0.4em\relax Paper - December, 2009
\bibitem{IEEEhowto:kopka}
Arthur T. Benjamin & Eric Huggins, \emph{Optimal BlackJack Strategy with "Lucky Bucks" }\hskip 1em plus
  0.5em minus 0.4em\relax Paper - Winter, 1993
\bibitem{IEEEhowto:kopka}
Don Schlesinger, \emph{Blackjack Attack: Playing the Pros' Way }\hskip 1em plus
  0.5em minus 0.4em\relax Paperback - June, 1997
\bibitem{IEEEhowto:kopka}
Kenneth R Smith, \emph{Blackjack Basic Strategy Chart: 4/6/8 Decks, Dealer Hits Soft 17 }\hskip 1em plus
  0.5em minus 0.4em\relax Cards - October, 2008
\end{thebibliography}

\end{document}


