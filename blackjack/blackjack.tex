\usepackage{datetime}
\documentclass[journal]{IEEEtran}
\hyphenation{op-tical net-works semi-conduc-tor}
\begin{document}
\title{Kunnen we we het casino verslaan?}
\author{Caudenberg~Samuel, Demeulemeester~Michiel, De~Wilde~Michiel, Verelst~Thomas}
\maketitle


\begin{abstract}
Tijdens ons project Onderzoekstechnieken gaan we op zoek naar de invloed op de winst bij een ronde BlackJack, naargelang het aantal spelers, waarbij 1 speler (die we aanschouwen als onszelf) gebruik maakt van de Card Counting Strategy en alle andere spelers de basic Strategy toepassen. De resultaten van dit onderzoek zullen aantonen wanneer de speler die gebruik maakt van de Card Counting Strategy een relatief hoge kans maakt om de ronde te winnen, of wanneer hij het net bij een kleine bet moet houden. Met behulp van enkele simulaties hebben we de nodige data verzameld, daarna hebben we deze geanalyseerd en uiteindelijk tonen we aan wat we onderzocht hebben door gebruik te maken van grafieken. 
\end{abstract}

\section{Voorwoord}
Om duidelijk te maken welke strategie�n we gaan gebruiken, zullen we deze beiden kort toelichten. 
Eerst hebben we de Basic Strategy (zie onderstaande tabel).

De betekenis van 'Stiff hand': Als de eerste kaarten een totaal geven van 12-16, is dit een Stiff hand'. Wanneer dit voor de speler het geval is, en ook de dealer heeft een Stiff hand (hand tussen 2 en 6), dan moet je 'Standen'. Heeft de dealer echter een 'Pat' ( hand tussen 7 en Ace), dan moet de speler 'Hitten'. 
'Doubling down hands' betekent het verdubbelen van je bet, en dit doet de speler wanneer zijn eerste kaarten een totaal geven tussen 9 en 11 punten.

De tweede strategie die we gebruiken tijdens dit onderzoek is de Card Counting Strategy. 
 
\hfill \today

\section{Aanpak}
\subsection{Strategi�n}
\subsubsection{Count Strategie}
\subsubsection{Betting Strategie}
\subsection{Simulatiesoftware}
\subsection{Uitwerking}

\section{Resultaten}

Grafieken & data

\section{Conclusies}

De conclusies


\begin{thebibliography}{1}

\bibitem{IEEEhowto:kopka}
H.~Kopka and P.~W. Daly, \emph{A Guide to \LaTeX}, 3rd~ed.\hskip 1em plus
  0.5em minus 0.4em\relax Harlow, England: Addison-Wesley, 1999.
\end{thebibliography}

\end{document}


