	\usepackage{datetime}
\documentclass[journal]{IEEEtran}
\hyphenation{op-tical net-works semi-conduc-tor}
\begin{document}
\title{Kunnen we we het casino verslaan?}
\author{Groep~29: Caudenberg~Samuel, Demeulemeester~Michiel, De~Wilde~Michiel, Verelst~Thomas}
\maketitle


\begin{abstract}
In deze paper toetsen we of we het casino kunnen verslaan in blackjack. We zullen gebruik maken van een strategie die kan uitgevoerd worden aan een echte speeltafel, zonder dat er heel complex rekenwerk aan te pas komt. We zullen de omstandigheden van een normale situatie zoveel mogelijk trachten te weerspiegelen. Kunnen we op een eerlijke manier het casino te slim af zijn?
\end{abstract}

\section{Voorwoord}
Tijdens ons project Onderzoekstechnieken gaan we op zoek naar enkele invloeden op de winst bij een ronde BlackJack, naargelang het aantal spelers, waarbij 1 speler (die we aanschouwen als onszelf) gebruik maakt van de Card Counting Strategy en alle andere spelers de basic Strategy toepassen. De resultaten van dit onderzoek zullen aantonen dat de speler die gebruik maakt van de Card Counting Strategy een relatief hoge kans maakt om de ronde te winnen ten opzichte van de spelers die de basic strategy toepassen. Met behulp van enkele simulaties hebben we de nodige data verzameld, daarna hebben we deze geanalyseerd en uiteindelijk kunnen dan conclusies trekken met behulp van enkele tabellen of grafieken.

\section{Aanpak}
Voor ons onderzoek moeten we enkele simulaties maken in verschillende situaties. Als we dit niet zouden doen dan zouden we onze hypothese niet kunnen bewijzen.
Bij deze simulaties

\subsection{Strategie\"en}

Wanneer je blackjack speelt kun je best vertrouwen op een vast schema of een strategie. Dit zorgt ervoor dat je je resultaten gemakkelijk kunt evalueren.
Door te experimenteren met de gekozen strategie kan men dus een beter resultaat bekomen. Deze strategie moet dan wel veelvuldig getest worden.

Basic Strategy is de basisstrategie bij uitstek. Beginnende blackjackspelers gaan deze strategie gebruiken wanneer ze ter ontspanning in een casino spelen. Het gevoel dat je een bestaande en bewezen strategie gebruikt is toch een geruststalling voor veel spelers.
Bij basic strategy wordt geen rekening gehouden met de card count van de tafel.

Red7shoeH17DASI18 is een strategie die gebruikt maakt van card counting. Deze strategie zorgt ervoor dat je een beter idee krijgt wanneer je hoger moet inzetten.
Verdere uitleg over de count strategie vind je verder in de paper.
\newline
\subsubsection{Basic Strategie}

Raadplaag tabel \ref{table:basic} achteraan in de paper.

De betekenis van 'Stiff hand': Als de eerste kaarten een totaal geven van 12-16, is dit een Stiff hand'. Wanneer dit voor de speler het geval is, en ook de dealer heeft een Stiff hand (hand tussen 2 en 6), dan moet je 'Standen'. Heeft de dealer echter een 'Pat' ( hand tussen 7 en Ace), dan moet de speler 'Hitten'. 
'Doubling down hands' betekent het verdubbelen van je bet, en dit doet de speler wanneer zijn eerste kaarten een totaal geven tussen 9 en 11 punten.
\newline
\subsubsection{Count Strategie}

Bij Card Counting wordt rekening gehouden met de kaarten die al gespeeld zijn. Doordat men deze kaarten al weet kan men de kans inschatten dat andere kaarten gaan vallen tijdens het spel. Dit zorgt voor een voordeel voor de speler. Casino-spellen zijn meestal in het nadeel van de speler maar door deze strategi�n goed toe te passen kan men het verlies beperken of zelfs winst maken.
Alle kaarten krijgen een bepaalde waarde toegewezen en wanneer de kaarten voorkomen in het spel wordt een centrale count bijgehouden.
Deze centrale count geeft de speler een idee welke kaarten de grootste kans hebben om te vallen. Dit is een belangerijk voordeel voor de speler.
\newline
\subsubsection{Betting Strategie}

Aan de hand van de count strategie kan men de inzet per spel aanpassen. Als er een grotere kans is dat je 21 kunt hebben dan zal je een hogere inzet plaatsen. Spelers die basic strategy gebruiken doen aan flat betting. Dit betekent dat ze altijd dezelfde hoeveelheid inzetten. Je inzet aanpassen is ook ��n van de belangerijkste factoren om je winstgevendheid te onderhouden.

\subsection{Simulatiesoftware}

Voor dit onderzoek hebben we bestaande software gebruikt, meerbepaald powersim. Powersim geeft de onderzoeker de mogelijkheid om veel situaties te simuleren.
Het testen van de strategie�n en situaties in een casino zou onmogelijk zijn. De hoeveelheid van handen, die gesimuleerd moeten worden, zou veel te veel geld kosten.
Het formaat waarin de resultaten worden opslagen liet toch wel we de wensen over.
Dit omzetten naar csv formaat bleek toch niet zo gemakkelijk.

\subsection{Uitwerking}

Voor het onderzoek hebben we vier tafels van 10.000.000 handen gesimuleerd. Er zijn tafels van 1, 2, 3 en 4 personen. De persoon, die de card counting strategie gebruikt, is altijd de eerste speler. De andere spelers maken gebruik van basic strategy.
Doordat de andere spelers basic strategy gebruiken en dus geen card count bijhouden, kunnen we ze alleen maar dezelfde inzet laten gebruiken.

\section{Resultaten}
\subsection{Aantal spelers als variabele}
We bekijken eerst wat de invloed is van het aantal spelers aan de tafel.
Onze speler gebruikt een countstrategie, terwijl de andere spelers aan zijn tafel een gewone basic strategy toepassen. Het startkrediet is telkens \textdollar 10.000 en er worden telkens 10.000.000 rondes gespeeld. In tabel \ref{table:spelers} wordt het resterende krediet ook in percent ten op zichte van het startkrediet getoond, zodat duidelijk is dat het verschil minimaal is.

\begin{table}[h]
\normalsize
\caption{Krediet na 10.000.000 rondes met \textdollar 10.000 startkrediet}
\label{table:spelers}
\begin{center}
\def\arraystretch{1.5}
\begin{tabular}{c|c c c c}
Aantal spelers & 1 & 2 & 3 & 4\\
\hline
Krediet & \textdollar 96,78 & \textdollar 107,77 & \textdollar 92,82 & \textdollar 111,27 \\
Percent & 0,9678\% & 1,077\% & 0,9282\% & 1,1127\% \\
\end{tabular}
\end{center}
\end{table}

\subsection{Vergelijking met de andere spelers}
Mits het duidelijk is dat er toch nog verlies wordt gemaakt, vergelijken we nu de resultaten van al de spelers aan een tafel. We maken hier enkel de vergelijking wanneer er vier spelers aan de tafel zitten, want in dat scenario had de speler die de cardcounting strategie gebruikt uiteindelijk het minste verlies.

Dezelfe regels gelden als in voorgaande sectie. Speler 1 speelt dus met een count strategie, en de andere spelers aan de tafel met een gewone basic strategy. De spelers beginnen allen met \textdollar 10.000 startkrediet en er worden 10.000.000 rondes gespeeld. Een negatief bedrag betekent dat de speler al zijn startkrediet heeft opgespeeld en dus extra krediet ter beschikking heeft gekregen.

We introduceren in tabel \ref{table:vierspelers} nu ook het begrip \textit{risk of ruin}. Dit is het percentage waarmee aangeduid wordt hoe groot de kans is dat iemand al zijn krediet verliest. De \textit{risk of ruin} hangt dan ook af van hoeveel ingezet wordt en hoeveel keren gespeeld wordt.

\begin{table}[h]
\normalsize
\caption{Vergelijking van de spelers na 10.000.000 rondes met \textdollar 10.000 startkrediet}
\label{table:vierspelers}
\begin{center}
\def\arraystretch{1.5}
\begin{tabular}{c|c c c c}
Speler & 1 & 2 & 3 & 4\\
\hline
Krediet & \textdollar 111,27 & \textdollar -33,24 & \textdollar -35,68 & \textdollar -35,67 \\
Percent & 1,1127\% & -0.3324\% & -0,3568\% & -0,3567\% \\
Risk of ruin & 13,53\% & 100\% & 100\% & 100\% \\
\end{tabular}
\end{center}
\end{table}

\section{Conclusies}

Uit de eerste tabel kunnen we concluderen dat het aantal spelers in dit specifieke geval, enkel 1 speler die card counting toepast, geen grote invloed heeft op ons resultaat. Dit komt waarschijnlijk door de betting strategie die we hebben besloten toe te passen. Deze zal kijken naar onze winstkansen en een percentage van ons van het resterende krediet inzetten. Het resterende krediet na 10 milioen keer spelen schommelt telkens tussen 92,82 (bij 3 spelers) en 111,27 (bij 4 spelers). In geen enkel geval loopt de speler het gevaar om al zijn krediet kwijt te spelen, in alle gevallen blijf enkel nog ongeveer 1 percent van het startkrediet over.

In het geval van 4 tegenstanders zien we dat enkel de speler die cardcounting toepast nog krediet over heeft, de rest van de spelers die basic strategy toepassen maken allemaal verlies.

We kunnen dus afleiden dat je op lange termijn altijd verlies zult maken, ookal pas je een superieure strategie toe ten opzichte van de andere spelers. Natuurlijk kan je als je minder spelletjes speelt en geluk hebt wel winst maken op korte termijn, maar dit is zeker geen goede mannier om geld te gaan verdienen in casino's


\begin{thebibliography}{1}
\bibitem{IEEEhowto:kopka}
Ariell Zimran, Anna Klis, Alejandra Fuster and Christopher Rivelli, \emph{The Game of Blackjack and Analysis of Counting Cards }\hskip 1em plus 0.5em minus 0.4em\relax Paper - December, 2009
\bibitem{IEEEhowto:kopka}
Arthur T. Benjamin and Eric Huggins, \emph{Optimal BlackJack Strategy with "Lucky Bucks" }\hskip 1em plus
  0.5em minus 0.4em\relax Paper - Winter, 1993
\bibitem{IEEEhowto:kopka}
Don Schlesinger, \emph{Blackjack Attack: Playing the Pros' Way }\hskip 1em plus
  0.5em minus 0.4em\relax Paperback - June, 1997
\bibitem{IEEEhowto:kopka}
Kenneth R Smith, \emph{Blackjack Basic Strategy Chart: 4/6/8 Decks, Dealer Hits Soft 17 }\hskip 1em plus
  0.5em minus 0.4em\relax Cards - October, 2008
\bibitem{IEEEhowto:kopka}
\emph{High-Low Card Counting Strategy Introduction}\hskip 1em plus
  0.5em minus 0.4em\relax  geraadpleegd op 2 april 2015 via {http://wizardofodds.com/games/blackjack/card-counting/high-low/}
\bibitem{IEEEhowto:kopka}
\emph{How to use blackjack strategy trainer}\hskip 1em plus
  0.5em minus 0.4em\relax 2 geraadpleegd op april 2015 via {http://blackjackisonline.com/blackjack-strategy-trainer.aspx}
\bibitem{IEEEhowto:kopka}
\emph{Release of Blackjack PowerSim Open Source Card Counting Simulation Software}\hskip 1em plus
  0.5em minus 0.4em\relax geraadpleegd op 2 april 2015 via {http://www.blackjackforumonline.com/content/SimSimp\_Beta.htm}
	
\end{thebibliography}

\begin{multicols}
\begin{table}[p]
\normalsize
\caption{Basic Strategy}
\label{table:basic}
\begin{center}
\def\arraystretch{1.5}
\begin{tabular}{c| l | l}
\textbf{Type of hand} & \textbf{Description} & \textbf{Basic strategy}\\
\hline
Soft hands & Card combination includes and ace &  Stand if \geq 18\\
& &  Hit if \leq17\\
	\hline
Hard hands & Card combination includes no ace &  Hit if \leq 8 \\
& Ace in hand counts as 1 point &  Stand if \geq17\\
& & Stiff hands if between 12\& 16\\
& & Doubling down hands if between 9 \& 11\\
\hline
Splitting pairs  & Card combination includes & Always split if aces\\
hands & the cards of equal rank & Never split if 5 \& 10\\

\end{tabular}
\end{center}
\end{table}
\end{multicols}

\end{document}